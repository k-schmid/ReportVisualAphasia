% 
% Annual Cognitive Science Conference
% Sample LaTeX Paper -- Proceedings Format
% 

% Original : Ashwin Ram (ashwin@cc.gatech.edu)       04/01/1994
% Modified : Johanna Moore (jmoore@cs.pitt.edu)      03/17/1995
% Modified : David Noelle (noelle@ucsd.edu)          03/15/1996
% Modified : Pat Langley (langley@cs.stanford.edu)   01/26/1997
% Latex2e corrections by Ramin Charles Nakisa        01/28/1997 
% Modified : Tina Eliassi-Rad (eliassi@cs.wisc.edu)  01/31/1998
% Modified : Trisha Yannuzzi (trisha@ircs.upenn.edu) 12/28/1999 (in process)
% Modified : Mary Ellen Foster (M.E.Foster@ed.ac.uk) 12/11/2000
% Modified : Ken Forbus                              01/23/2004
% Modified : Eli M. Silk (esilk@pitt.edu)            05/24/2005
% Modified : Niels Taatgen (taatgen@cmu.edu)         10/24/2006
% Modified : David Noelle (dnoelle@ucmerced.edu)     11/19/2014

%% Change "letterpaper" in the following line to "a4paper" if you must.

\documentclass[10pt,letterpaper]{article}

\usepackage{cogsci}
\usepackage{pslatex}
\usepackage{apacite}
\usepackage{graphicx}

\graphicspath{{./Figures/}}

\title{Transfer learning across different sensory modalities in Brain Computer Interfaces}
 
\author{{\large \bf Konstantin Schmid (konstantin.schmid@uranus.uni-freiburg.de)} \\
	  Center for Cognitive Science Institute of Computer Science and Social Research,\\
	  Friedrichstr. 50 D-79098 Freiburg, Germany\\}


\begin{document}

\maketitle


\begin{abstract}
The abstract should be one paragraph, indented 1/8~inch on both sides,
in 9~point font with single spacing. The heading ``{\bf Abstract}''
should be 10~point, bold, centered, with one line of space below
it. This one-paragraph abstract section is required only for standard
six page proceedings papers. Following the abstract should be a blank
line, followed by the header ``{\bf Keywords:}'' and a list of
descriptive keywords separated by semicolons, all in 9~point font, as
shown below.

\textbf{Keywords:} 
add your choice of indexing terms or keywords; kindly use a
semicolon; between each term
\end{abstract}

\section{Introduction}
The P300 is a prominent event-related brain potential (ERP) which is relatively easy and reliable to evoke and thus highly interesting in brain computer interface research.
It was first described by \citeA{Sutton_1965} as a positive-going component with a peak amplitude around 300~ms after the eliciting stimulus onset.
These stimuli most commonly occur in an oddball paradigm where the subject is instructed to respond (mentally or physically) to a rare target event in a sequence of frequent non-target events.
The P300, also called late positive component (LPC), is seen to reflect information-processing activity instead of primary sensory activity and is thus less dependent on stimulus modality \cite{Wolpaw_2012}.
Such amodal ERPs are referred to as endogenous components.
Several studies propose that endogenous components like the P300 occur independent of modality if the role of the eliciting stimulus stays the same \cite{Chang_2013,Ji_1999,Sangal_1996,Polich_2007}.
Some results even suggest a comparably similar shape, topography and timing of the ERP if relevant stimulus features (discriminability, timings, probability) are kept same.

The majority of brain computer interfaces (BCI) developed in recent decades rely on unimodal approaches \cite{Wolpaw_2012}.
However the users' special needs and the idea of a user-centered BCI can necessitate a multi-modal design.
There are different case studies requesting a BCI design which easily allows the change of modality if for example paradigms of one specific modality cannot be controlled by the user \cite{Schreuder_2013, Kaufmann_2013}.
As up to present most BCIs rely on supervised learning \cite{Wolpaw_2012} a calibration of the system is needed to produce reliable output.
To achieve a minimum of usability recalibration times after modality changes must be kept short and thus a transfer of the calibrated system to another modality would be demandable. 
We evaluated the transfer 


Although the P300 typically peaks around 300~ms after stimulus onset the latency is highly dependent on the experimental design.
The less distinctive the target stimulus is for example the longer the time to detect and evaluate it and thus the longer the latency is \cite{Polich_2007}.
 
Later evaluations, however, suggested the latency to be highly dependent on the experimental design and may vary from 250 to 750 ms \cite{Wolpaw_2012}.

\section{Methods}

\subsection{Subjects}
One pilot session was performed of which the data was used to fine-tune experimental parameters but was excluded from further analysis.
Five healthy adult students (two male, three female, mean age xx, SD $\pm$ xx, range xx-26, advanced level of German) participated in the study.
All participants had normal or corrected to normal vision.

\subsection{Experimental Setup}
Subjects were seated in front of computer monitor (size xx'', distance 80\ cm) and were surrounded by a ring of six speakers equally distributed around the participant (distance 60 cm, AMUSE paradigm \cite{Schreuder_2010}). TODO CHECK
The stimuli were presented either visually on the monitor or auditory by means of the speakers.
 
\subsubsection{Data acquisition}
The EEG was recorded using an electrode cap with 63 passive Ag/AgCl electrodes (EasyCap GmbH).
Signals were amplified using a BrainAmp Amplifier (Brainproducts GmbH), sampled at 1 kHz and band-filtered between xx. TODO: CHECK
The electrodes were placed according to the 10-20-system, referenced against the nose and grounded with a forehead electrode.
EOG was recorded with an additional channel below the right eye.
The impedances were kept below 10 k$\Omega$.

\subsubsection{Stimuli}
The stimuli consist of six sentence-word pairs which were learned by the subjects during a familiarization phase:

\begin{tabular}[t]{ll}
Die Tonerkartusche steckt schon im 	& ... Drucker.\\
In der Getr\"ankekiste steht noch eine volle & ... Flasche.\\
Am Ende der großen Pause l\"autet die & ... Glocke.\\
Die Jacke des Kapit\"ans hat goldene & ... Kn\"opfe.\\
Zur Beglaubigung benutzt die \\
Sachbearbeiterin einen amtlichen	& ... Stempel.\\
Zum Nachf\"ullen des Motor\"ols nimmt\\
man am besten einen & ... Trichter.\\
\end{tabular}

At the beginning of each experimental trial one of the six sentences was presented from one loudspeaker.
The last word was missing indicating the target word.
Subjects were instructed to focus their attention on the target word presentation in the following trial.

\subsubsection{Experimental Structure}
The Experiment was split into three blocks.
Each block contained six runs, with alternating modality (visual vs. auditory), meaning the target and non-target words were presented via the screen or the loudspeakers respectively.
In each run every end-of-sentence word was target once leading to six trials per run.
A trial consists of 15 presentation of each word thus 90 stimuli were presented per trial (15 target 75 non-target).
Stimulus onset asynchrony (SOA) was kept at 250 ms.

In visual runs the target sentence was cued always from the same loudspeaker (left frontal).
Afterwards the six end-of-sentence words were presented on the screen as images in a 2 x 3 matrix (see Figure \ref{fig:visual_stimuli}).
\begin{figure}[ht]
	\includegraphics[width=\linewidth]{fig_visual_stimuli}
	\caption{Visual stimulus presentation.} 
	\label{fig:visual_stimuli}
\end{figure} TODO CHANGE IMAGE
After a short target image detection pause of 4 seconds the stimulation started.
Therefor images were highlighted individually by brightness enhancement, rotation, enlargement and a trichromatic grid overlay which was suggested to produce a strong visual ERP by \citeA{Tangermann_2011}.
The stimulus duration was 250 ms, thus stimulations took place with zero inter-stimulus intervals.

In auditory runs there was a one-to-one mapping between speakers and words.
Sentences were presented from the same speaker as the target word indicating the direction of the current trial which was as well cued by a screen animation.
Stimulus duration of the six words was 300 ms hence stimuli slightly overlapped by 50 ms.


\bibliographystyle{apacite}

\setlength{\bibleftmargin}{.125in}
\setlength{\bibindent}{-\bibleftmargin}

\bibliography{CogSci_Template}


\end{document}
